\documentclass{article}
\usepackage[margin=0.7in]{geometry}
\usepackage{amsmath}
\usepackage{cite} 
\usepackage{array}
\usepackage{graphicx}
\usepackage{tcolorbox}
\usepackage{xcolor}
\usepackage{engord}
\usepackage{bm}
\usepackage{booktabs}
\usepackage{pdfpages}
\usepackage{amssymb}
% \usepackage{algorithmic}
% \usepackage{algorithm}
\usepackage{multirow}
\usepackage{longtable}

\let\oldemptyset\emptyset
\let\emptyset\varnothing
\newcommand{\mb}[1]{\mathbf{#1}}
\newcommand{\mbg}[1]{\boldsymbol{#1}}
\newcommand{\minimize}[1]{\underset{{#1}}{\text{minimize}}}
\newcommand{\maximize}[1]{\underset{{#1}}{\text{maximize}}}
\newcommand{\sminimize}[1]{\underset{{#1}}{\text{min}}}
\newcommand{\saximize}[1]{\underset{{#1}}{\text{max}}}
\newcommand{\st}{\text{subject to}}


\title{\Large \bf Synthesizing Grid Data with Cyber Resilience and Privacy Guarantees \\ \vspace{0.5cm} \large Online Appendix}
\author{Shengyang Wu, Vladimir Dvorkin}
\date{}


\begin{document}
\maketitle
\section{Reformulation of (7)}

The original formulation of (7) in the manuscript are as follows

\begin{subequations}\label{prob:RO}
\begin{align}
    C_{\text{att}}^{\text{RO}}(\mb{d})=&\;\minimize{\mb{x}}\;\; \mb{c}^{\top}\mb{x}\\
    \st\;\;\;& \underset{\mbg{\delta_k\in\Delta}}{\max}\left[\mb{a}_{k}^{\top}\mb{x} + \mb{b}_{k}^{\top}(\mb{d}+\bm{\delta}_k) + e_{k}\right] \leqslant \mb{0},\forall k,\!\!\! \label{eq:RO st}
\end{align}
\end{subequations}

Constraint (\ref{eq:RO st}) can be rewritten as:
\begin{align}
    \left[
    \begin{array}{l}
        \underset{\delta}{\text{max}}\quad \mb{b}_{k}^{\top}\mbg{\delta}_{k}   \\
        \;\;\text{s.t.}\quad \underline{\mbg{\delta}} \leqslant \mbg{\delta}_{k} \leqslant \overline{\mbg{\delta}} \quad :\overline{\mbg{\mu}}_k,\underline{\mbg{\mu}}_k\\
        \quad\quad\quad \bm{1}^{\top}\mbg{\delta}_{k}=0 \quad :\lambda_k
    \end{array}
    \right]
    \leqslant -\mb{a}_{k}^{\top}\mb{x}-\mb{b}_k^{\top}\mb{d}-e_{k}\quad \forall k=1,\dots,K
    \label{eq:Princeton_BkConstraint}
\end{align}

We can get the Lagrangian function of the left-hand side problem as 

\begin{align}
    \mathcal{L}(\overline{\mbg{\mu}}_k,\underline{\mbg{\mu}}_k,\lambda_k)&=-\mb{b}_{k}^{\top}\mbg{\delta}_{k}+\underline{\mbg{\mu}}_k^{\top}(-\mbg{\delta}_{k}+\underline{\mbg{\delta}})+\overline{\mbg{\mu}}_k^{\top}(\mbg{\delta}_{k}-\overline{\mbg{\delta}})+\lambda_k^{\top}\bm{1}^{\top}\mbg{\delta}_{k} \\ \nonumber
    &=(-\mb{b}_{k}^{\top}-\underline{\mbg{\mu}}_k^{\top}+\overline{\mbg{\mu}}_k^{\top}+\lambda_k^{\top}\bm{1}^{\top})\mbg{\delta}_{k} +\underline{\mbg{\mu}}_k^{\top}\underline{\mbg{\delta}}-\overline{\mbg{\mu}}_k^{\top}\overline{\mbg{\delta}}
\end{align}

The dual of problem for the left-hand side maximization problem (\ref{eq:Princeton_BkConstraint}) can be expressed as:

\begin{align}
    \underset{\overline{\mbg{\mu}}_k,\underline{\mbg{\mu}}_k,\lambda_k}{\text{maximize}}\quad& \underline{\mbg{\mu}}_k^{\top}\underline{\mbg{\delta}}-\overline{\mbg{\mu}}_k^{\top}\overline{\mbg{\delta}} \label{eq:Printon_Dual} \\ \nonumber
    \text{subject to} \quad& -\mb{b}_{k}-\underline{\mbg{\mu}}_k+\overline{\mbg{\mu}}_k+\bm{1}\cdot\lambda_k=0 \\ \nonumber
    \quad& \underline{\mbg{\mu}}_k,\overline{\mbg{\mu}}_k \geqslant 0,\;\lambda_k \in \text{FREE} 
\end{align} 
Take (\ref{eq:Printon_Dual}) into (\ref{eq:Princeton_BkConstraint}), the whole formulation can be written as:
\begin{subequations}
\begin{align}
    \underset{\mb{x},\overline{\mbg{\mu}},\underline{\mbg{\mu}},\lambda}{\text{minimize}}\quad& \mb{c}^{\top}\mb{x}\\
    \text{subject to}\quad& \overline{\mbg{\mu}}_k^{\top}\overline{\mbg{\delta}}-\underline{\mbg{\mu}}_k^{\top}\underline{\mbg{\delta}} \leqslant-\mb{a}_{k}^{\top}\mb{x}-\mb{b}_k^{\top}\mb{d}-e_{k} \quad \forall k=1,\dots,K\\
    \quad&\mb{b}_{k}-\overline{\mbg{\mu}}_k+\underline{\mbg{\mu}}_k-\bm{1}\cdot\lambda_k=\bm{0}\quad \forall k=1,\dots,K\\
    \quad& \underline{\mbg{\mu}}_k,\overline{\mbg{\mu}}_k \geqslant 0,\;\lambda_k \in \text{FREE}\\
\end{align}
\label{eq:Princeton_BkDelta}
\end{subequations}
which is the formulation (8) in the manuscript.

\end{document}